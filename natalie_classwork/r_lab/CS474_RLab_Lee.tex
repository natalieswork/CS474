% Options for packages loaded elsewhere
\PassOptionsToPackage{unicode}{hyperref}
\PassOptionsToPackage{hyphens}{url}
%
\documentclass[
]{article}
\usepackage{amsmath,amssymb}
\usepackage{iftex}
\ifPDFTeX
  \usepackage[T1]{fontenc}
  \usepackage[utf8]{inputenc}
  \usepackage{textcomp} % provide euro and other symbols
\else % if luatex or xetex
  \usepackage{unicode-math} % this also loads fontspec
  \defaultfontfeatures{Scale=MatchLowercase}
  \defaultfontfeatures[\rmfamily]{Ligatures=TeX,Scale=1}
\fi
\usepackage{lmodern}
\ifPDFTeX\else
  % xetex/luatex font selection
\fi
% Use upquote if available, for straight quotes in verbatim environments
\IfFileExists{upquote.sty}{\usepackage{upquote}}{}
\IfFileExists{microtype.sty}{% use microtype if available
  \usepackage[]{microtype}
  \UseMicrotypeSet[protrusion]{basicmath} % disable protrusion for tt fonts
}{}
\makeatletter
\@ifundefined{KOMAClassName}{% if non-KOMA class
  \IfFileExists{parskip.sty}{%
    \usepackage{parskip}
  }{% else
    \setlength{\parindent}{0pt}
    \setlength{\parskip}{6pt plus 2pt minus 1pt}}
}{% if KOMA class
  \KOMAoptions{parskip=half}}
\makeatother
\usepackage{xcolor}
\usepackage[margin=1in]{geometry}
\usepackage{graphicx}
\makeatletter
\def\maxwidth{\ifdim\Gin@nat@width>\linewidth\linewidth\else\Gin@nat@width\fi}
\def\maxheight{\ifdim\Gin@nat@height>\textheight\textheight\else\Gin@nat@height\fi}
\makeatother
% Scale images if necessary, so that they will not overflow the page
% margins by default, and it is still possible to overwrite the defaults
% using explicit options in \includegraphics[width, height, ...]{}
\setkeys{Gin}{width=\maxwidth,height=\maxheight,keepaspectratio}
% Set default figure placement to htbp
\makeatletter
\def\fps@figure{htbp}
\makeatother
\setlength{\emergencystretch}{3em} % prevent overfull lines
\providecommand{\tightlist}{%
  \setlength{\itemsep}{0pt}\setlength{\parskip}{0pt}}
\setcounter{secnumdepth}{-\maxdimen} % remove section numbering
\ifLuaTeX
  \usepackage{selnolig}  % disable illegal ligatures
\fi
\IfFileExists{bookmark.sty}{\usepackage{bookmark}}{\usepackage{hyperref}}
\IfFileExists{xurl.sty}{\usepackage{xurl}}{} % add URL line breaks if available
\urlstyle{same}
\hypersetup{
  pdftitle={R Lab Assignment: Field Trip to Yellowstone Park},
  pdfauthor={Natalie Lee},
  hidelinks,
  pdfcreator={LaTeX via pandoc}}

\title{R Lab Assignment: Field Trip to Yellowstone Park}
\author{Natalie Lee}
\date{Fall 2023}

\begin{document}
\maketitle

\hypertarget{introduction}{%
\section{Introduction}\label{introduction}}

Yellowstone National Park has a popular tourist attraction called Old
Faithful. Old Faithful is a hot water geyser that has regular,
spectacular eruptions that can last minutes.

The goal of this analysis is to provide valuable insight into the
behavior of the geyser that serves to help people who are visiting the
attraction get the most out of their experience (i.e.~spend less time
looking at a non-erupting hole in the ground).

To analyze the patterns of Old Faithful's eruptions we will be using the
\texttt{faithful.csv} data set. In the data set, 272 cases of eruptions
are examined. There are two observation variables. The first one, called
eruptions, is the duration in minutes of the geyser eruption. The
eruption duration in this data set range from 1.6 to 5.1 minutes! The
second variable, called waiting, is the length of the subsequent waiting
period in minutes until the next eruption. These waiting periods range
from 43 to 96 minutes.

In the following sections, various statistical analysis and
visualization methods will be utilized to explore the waiting periods in
this data set.

\hypertarget{frequency-distribution-of-eruption-waiting-periods}{%
\section{Frequency Distribution of Eruption Waiting
Periods}\label{frequency-distribution-of-eruption-waiting-periods}}

Frequency distribution is used to see the number of a data value in a
given interval. In the case of waiting time, the frequency distribution
can be used to count how many eruptions in the data set occurred at
after specific time interval.

In the table below, the first column indicates the time interval in
minutes, and the second column indicates the frequency of data points
that occurred in that interval. For example, at the 40 to 50 minute time
interval, 21 of the subsequent eruptions began.

\begin{verbatim}
##          waiting.freq
## [40,50)            21
## [50,60)            56
## [60,70)            26
## [70,80)            77
## [80,90)            80
## [90,100)           12
\end{verbatim}

\hypertarget{duration-sub-interval-with-the-most-eruptions}{%
\section{Duration Sub-Interval with the Most
Eruptions}\label{duration-sub-interval-with-the-most-eruptions}}

The table in the previous section can provide information about which
waiting time interval most of the eruptions occur after. The figure of
``{[}80,90)'' below indicates that most of the eruptions occur between
80 to 90 minutes after the previous eruption.

Knowing this, a tourist can expect that most likely an eruption will
take place about an hour and twenty minutes to an hour and a half after
the previous eruption. That being said, if one is just arriving after a
eruption, they should feel free to take a detour to the gift shop.

\begin{verbatim}
## [1] "[80,90)"
\end{verbatim}

\hypertarget{histogram-of-eruption-waiting-period}{%
\section{Histogram of Eruption Waiting
Period}\label{histogram-of-eruption-waiting-period}}

To get a more visual look at the frequency distribution of the waiting
periods, a histogram is helpful. The figure below shows on the x-axis
the waiting time interval and the y-axis indicate the frequency of data
points in that time interval.

In this way of looking at the data, the peak with the highest frequency
is above the range we indicated previous at 80 to 90 minutes.
Additionally, a second sub-interval of time in which eruptions occur the
most is around 50-60 minutes after an eruption becomes clear.

\includegraphics{CS474_RLab_Lee_files/figure-latex/hist-1.pdf}

\hypertarget{relative-frequency-distribution-of-eruption-waiting-periods}{%
\section{Relative Frequency Distribution of Eruption Waiting
Periods}\label{relative-frequency-distribution-of-eruption-waiting-periods}}

While knowing the count of occurrences of eruptions at specific waiting
periods is helpful, putting these values in a proportions can provide a
figure that is more easily framed in the context of the entire data set.
To calculate these proportions the frequency at a specific time interval
was divided by the total count of eruptions.

In the table below, the first column indicates the time interval in
minutes, and the second column indicates the proportion of data points
that occurred in that interval. For example, at the 40 to 50 minute time
interval, 8\% of the subsequent eruptions began.

\begin{verbatim}
##          waiting.relfreq
## [40,50)             0.08
## [50,60)             0.21
## [60,70)             0.10
## [70,80)             0.28
## [80,90)             0.29
## [90,100)            0.04
\end{verbatim}

\hypertarget{cumulative-frequency-distribution-of-eruption-waiting-periods}{%
\section{Cumulative Frequency Distribution of Eruption Waiting
Periods}\label{cumulative-frequency-distribution-of-eruption-waiting-periods}}

Cumulative frequencies reveal the overall pattern of eruptions,
showcasing how the geyser's activity accumulates over time.

In the table below, the first column indicates the time interval in
minutes, and the second column indicates the number of eruptions that
have occurred up to that point. For example, at the 50 to 60 minute time
interval, 77 out of the 272 eruptions in the data set have happened.

\begin{verbatim}
##          waiting.cumfreq
## [40,50)               21
## [50,60)               77
## [60,70)              103
## [70,80)              180
## [80,90)              260
## [90,100)             272
\end{verbatim}

\hypertarget{cumulative-frequency-graph-of-eruption-waiting-periods}{%
\section{Cumulative Frequency Graph of Eruption Waiting
Periods}\label{cumulative-frequency-graph-of-eruption-waiting-periods}}

The graph below provides a visual of the table in the previous section.
This graph showcases well how most of the eruptions in the data set
accumulated around the waiting times from 70 to 90 minutes.

\includegraphics{CS474_RLab_Lee_files/figure-latex/cum-freq-1.pdf}

\hypertarget{stem-and-leaf-plot-of-eruption-waiting-periods}{%
\section{Stem-and-Leaf Plot of Eruption Waiting
Periods}\label{stem-and-leaf-plot-of-eruption-waiting-periods}}

In the plot below, each digit on the right hand side represents the last
digit of a data point for waiting times. The digits on the left hand
side are the stems of plot and represent all but the last digit of a
data point.

A stem-and-leaf plot can provide a visual of data density. If there are
many leaves for a single stem, it indicates a high frequency of
eruptions occurred after that waiting period. The figure below shows a
high density of waiting times with the `8' digit stem or the 80-89
minute waiting period range.

\begin{verbatim}
## 
##   The decimal point is 1 digit(s) to the right of the |
## 
##   4 | 3
##   4 | 55566666777788899999
##   5 | 00000111111222223333333444444444
##   5 | 555555666677788889999999
##   6 | 00000022223334444
##   6 | 555667899
##   7 | 00001111123333333444444
##   7 | 555555556666666667777777777778888888888888889999999999
##   8 | 000000001111111111111222222222222333333333333334444444444
##   8 | 55555566666677888888999
##   9 | 00000012334
##   9 | 6
\end{verbatim}

\hypertarget{scatter-plot-of-eruption-durations-and-waiting-intervals}{%
\section{Scatter Plot of Eruption Durations and Waiting
Intervals}\label{scatter-plot-of-eruption-durations-and-waiting-intervals}}

Waiting periods are impacted by the duration of eruption. The scatter
plot below examines the relationship between the duration of the
geyser's eruption on the x-axis and the subsequent waiting time on the
y-axis.

There is a positive linear relationship between the two variables,
meaning that when the duration of a eruption is longer the subsequent
wanting period is also longer.

Knowing this, tourist who have arrived after an eruptions with a longer
duration can expect to wait longer than is they have arrived after a
shorter eruption.

\includegraphics{CS474_RLab_Lee_files/figure-latex/scatter-1.pdf}

\end{document}
